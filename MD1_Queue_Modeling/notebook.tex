
% Default to the notebook output style

    


% Inherit from the specified cell style.




    
\documentclass[11pt]{article}

    
    
    \usepackage[T1]{fontenc}
    % Nicer default font (+ math font) than Computer Modern for most use cases
    \usepackage{mathpazo}

    % Basic figure setup, for now with no caption control since it's done
    % automatically by Pandoc (which extracts ![](path) syntax from Markdown).
    \usepackage{graphicx}
    % We will generate all images so they have a width \maxwidth. This means
    % that they will get their normal width if they fit onto the page, but
    % are scaled down if they would overflow the margins.
    \makeatletter
    \def\maxwidth{\ifdim\Gin@nat@width>\linewidth\linewidth
    \else\Gin@nat@width\fi}
    \makeatother
    \let\Oldincludegraphics\includegraphics
    % Set max figure width to be 80% of text width, for now hardcoded.
    \renewcommand{\includegraphics}[1]{\Oldincludegraphics[width=.8\maxwidth]{#1}}
    % Ensure that by default, figures have no caption (until we provide a
    % proper Figure object with a Caption API and a way to capture that
    % in the conversion process - todo).
    \usepackage{caption}
    \DeclareCaptionLabelFormat{nolabel}{}
    \captionsetup{labelformat=nolabel}

    \usepackage{adjustbox} % Used to constrain images to a maximum size 
    \usepackage{xcolor} % Allow colors to be defined
    \usepackage{enumerate} % Needed for markdown enumerations to work
    \usepackage{geometry} % Used to adjust the document margins
    \usepackage{amsmath} % Equations
    \usepackage{amssymb} % Equations
    \usepackage{textcomp} % defines textquotesingle
    % Hack from http://tex.stackexchange.com/a/47451/13684:
    \AtBeginDocument{%
        \def\PYZsq{\textquotesingle}% Upright quotes in Pygmentized code
    }
    \usepackage{upquote} % Upright quotes for verbatim code
    \usepackage{eurosym} % defines \euro
    \usepackage[mathletters]{ucs} % Extended unicode (utf-8) support
    \usepackage[utf8x]{inputenc} % Allow utf-8 characters in the tex document
    \usepackage{fancyvrb} % verbatim replacement that allows latex
    \usepackage{grffile} % extends the file name processing of package graphics 
                         % to support a larger range 
    % The hyperref package gives us a pdf with properly built
    % internal navigation ('pdf bookmarks' for the table of contents,
    % internal cross-reference links, web links for URLs, etc.)
    \usepackage{hyperref}
    \usepackage{longtable} % longtable support required by pandoc >1.10
    \usepackage{booktabs}  % table support for pandoc > 1.12.2
    \usepackage[inline]{enumitem} % IRkernel/repr support (it uses the enumerate* environment)
    \usepackage[normalem]{ulem} % ulem is needed to support strikethroughs (\sout)
                                % normalem makes italics be italics, not underlines
    

    
    
    % Colors for the hyperref package
    \definecolor{urlcolor}{rgb}{0,.145,.698}
    \definecolor{linkcolor}{rgb}{.71,0.21,0.01}
    \definecolor{citecolor}{rgb}{.12,.54,.11}

    % ANSI colors
    \definecolor{ansi-black}{HTML}{3E424D}
    \definecolor{ansi-black-intense}{HTML}{282C36}
    \definecolor{ansi-red}{HTML}{E75C58}
    \definecolor{ansi-red-intense}{HTML}{B22B31}
    \definecolor{ansi-green}{HTML}{00A250}
    \definecolor{ansi-green-intense}{HTML}{007427}
    \definecolor{ansi-yellow}{HTML}{DDB62B}
    \definecolor{ansi-yellow-intense}{HTML}{B27D12}
    \definecolor{ansi-blue}{HTML}{208FFB}
    \definecolor{ansi-blue-intense}{HTML}{0065CA}
    \definecolor{ansi-magenta}{HTML}{D160C4}
    \definecolor{ansi-magenta-intense}{HTML}{A03196}
    \definecolor{ansi-cyan}{HTML}{60C6C8}
    \definecolor{ansi-cyan-intense}{HTML}{258F8F}
    \definecolor{ansi-white}{HTML}{C5C1B4}
    \definecolor{ansi-white-intense}{HTML}{A1A6B2}

    % commands and environments needed by pandoc snippets
    % extracted from the output of `pandoc -s`
    \providecommand{\tightlist}{%
      \setlength{\itemsep}{0pt}\setlength{\parskip}{0pt}}
    \DefineVerbatimEnvironment{Highlighting}{Verbatim}{commandchars=\\\{\}}
    % Add ',fontsize=\small' for more characters per line
    \newenvironment{Shaded}{}{}
    \newcommand{\KeywordTok}[1]{\textcolor[rgb]{0.00,0.44,0.13}{\textbf{{#1}}}}
    \newcommand{\DataTypeTok}[1]{\textcolor[rgb]{0.56,0.13,0.00}{{#1}}}
    \newcommand{\DecValTok}[1]{\textcolor[rgb]{0.25,0.63,0.44}{{#1}}}
    \newcommand{\BaseNTok}[1]{\textcolor[rgb]{0.25,0.63,0.44}{{#1}}}
    \newcommand{\FloatTok}[1]{\textcolor[rgb]{0.25,0.63,0.44}{{#1}}}
    \newcommand{\CharTok}[1]{\textcolor[rgb]{0.25,0.44,0.63}{{#1}}}
    \newcommand{\StringTok}[1]{\textcolor[rgb]{0.25,0.44,0.63}{{#1}}}
    \newcommand{\CommentTok}[1]{\textcolor[rgb]{0.38,0.63,0.69}{\textit{{#1}}}}
    \newcommand{\OtherTok}[1]{\textcolor[rgb]{0.00,0.44,0.13}{{#1}}}
    \newcommand{\AlertTok}[1]{\textcolor[rgb]{1.00,0.00,0.00}{\textbf{{#1}}}}
    \newcommand{\FunctionTok}[1]{\textcolor[rgb]{0.02,0.16,0.49}{{#1}}}
    \newcommand{\RegionMarkerTok}[1]{{#1}}
    \newcommand{\ErrorTok}[1]{\textcolor[rgb]{1.00,0.00,0.00}{\textbf{{#1}}}}
    \newcommand{\NormalTok}[1]{{#1}}
    
    % Additional commands for more recent versions of Pandoc
    \newcommand{\ConstantTok}[1]{\textcolor[rgb]{0.53,0.00,0.00}{{#1}}}
    \newcommand{\SpecialCharTok}[1]{\textcolor[rgb]{0.25,0.44,0.63}{{#1}}}
    \newcommand{\VerbatimStringTok}[1]{\textcolor[rgb]{0.25,0.44,0.63}{{#1}}}
    \newcommand{\SpecialStringTok}[1]{\textcolor[rgb]{0.73,0.40,0.53}{{#1}}}
    \newcommand{\ImportTok}[1]{{#1}}
    \newcommand{\DocumentationTok}[1]{\textcolor[rgb]{0.73,0.13,0.13}{\textit{{#1}}}}
    \newcommand{\AnnotationTok}[1]{\textcolor[rgb]{0.38,0.63,0.69}{\textbf{\textit{{#1}}}}}
    \newcommand{\CommentVarTok}[1]{\textcolor[rgb]{0.38,0.63,0.69}{\textbf{\textit{{#1}}}}}
    \newcommand{\VariableTok}[1]{\textcolor[rgb]{0.10,0.09,0.49}{{#1}}}
    \newcommand{\ControlFlowTok}[1]{\textcolor[rgb]{0.00,0.44,0.13}{\textbf{{#1}}}}
    \newcommand{\OperatorTok}[1]{\textcolor[rgb]{0.40,0.40,0.40}{{#1}}}
    \newcommand{\BuiltInTok}[1]{{#1}}
    \newcommand{\ExtensionTok}[1]{{#1}}
    \newcommand{\PreprocessorTok}[1]{\textcolor[rgb]{0.74,0.48,0.00}{{#1}}}
    \newcommand{\AttributeTok}[1]{\textcolor[rgb]{0.49,0.56,0.16}{{#1}}}
    \newcommand{\InformationTok}[1]{\textcolor[rgb]{0.38,0.63,0.69}{\textbf{\textit{{#1}}}}}
    \newcommand{\WarningTok}[1]{\textcolor[rgb]{0.38,0.63,0.69}{\textbf{\textit{{#1}}}}}
    
    
    % Define a nice break command that doesn't care if a line doesn't already
    % exist.
    \def\br{\hspace*{\fill} \\* }
    % Math Jax compatability definitions
    \def\gt{>}
    \def\lt{<}
    % Document parameters
    \title{MD1\_Simulation}
    
    
    

    % Pygments definitions
    
\makeatletter
\def\PY@reset{\let\PY@it=\relax \let\PY@bf=\relax%
    \let\PY@ul=\relax \let\PY@tc=\relax%
    \let\PY@bc=\relax \let\PY@ff=\relax}
\def\PY@tok#1{\csname PY@tok@#1\endcsname}
\def\PY@toks#1+{\ifx\relax#1\empty\else%
    \PY@tok{#1}\expandafter\PY@toks\fi}
\def\PY@do#1{\PY@bc{\PY@tc{\PY@ul{%
    \PY@it{\PY@bf{\PY@ff{#1}}}}}}}
\def\PY#1#2{\PY@reset\PY@toks#1+\relax+\PY@do{#2}}

\expandafter\def\csname PY@tok@w\endcsname{\def\PY@tc##1{\textcolor[rgb]{0.73,0.73,0.73}{##1}}}
\expandafter\def\csname PY@tok@c\endcsname{\let\PY@it=\textit\def\PY@tc##1{\textcolor[rgb]{0.25,0.50,0.50}{##1}}}
\expandafter\def\csname PY@tok@cp\endcsname{\def\PY@tc##1{\textcolor[rgb]{0.74,0.48,0.00}{##1}}}
\expandafter\def\csname PY@tok@k\endcsname{\let\PY@bf=\textbf\def\PY@tc##1{\textcolor[rgb]{0.00,0.50,0.00}{##1}}}
\expandafter\def\csname PY@tok@kp\endcsname{\def\PY@tc##1{\textcolor[rgb]{0.00,0.50,0.00}{##1}}}
\expandafter\def\csname PY@tok@kt\endcsname{\def\PY@tc##1{\textcolor[rgb]{0.69,0.00,0.25}{##1}}}
\expandafter\def\csname PY@tok@o\endcsname{\def\PY@tc##1{\textcolor[rgb]{0.40,0.40,0.40}{##1}}}
\expandafter\def\csname PY@tok@ow\endcsname{\let\PY@bf=\textbf\def\PY@tc##1{\textcolor[rgb]{0.67,0.13,1.00}{##1}}}
\expandafter\def\csname PY@tok@nb\endcsname{\def\PY@tc##1{\textcolor[rgb]{0.00,0.50,0.00}{##1}}}
\expandafter\def\csname PY@tok@nf\endcsname{\def\PY@tc##1{\textcolor[rgb]{0.00,0.00,1.00}{##1}}}
\expandafter\def\csname PY@tok@nc\endcsname{\let\PY@bf=\textbf\def\PY@tc##1{\textcolor[rgb]{0.00,0.00,1.00}{##1}}}
\expandafter\def\csname PY@tok@nn\endcsname{\let\PY@bf=\textbf\def\PY@tc##1{\textcolor[rgb]{0.00,0.00,1.00}{##1}}}
\expandafter\def\csname PY@tok@ne\endcsname{\let\PY@bf=\textbf\def\PY@tc##1{\textcolor[rgb]{0.82,0.25,0.23}{##1}}}
\expandafter\def\csname PY@tok@nv\endcsname{\def\PY@tc##1{\textcolor[rgb]{0.10,0.09,0.49}{##1}}}
\expandafter\def\csname PY@tok@no\endcsname{\def\PY@tc##1{\textcolor[rgb]{0.53,0.00,0.00}{##1}}}
\expandafter\def\csname PY@tok@nl\endcsname{\def\PY@tc##1{\textcolor[rgb]{0.63,0.63,0.00}{##1}}}
\expandafter\def\csname PY@tok@ni\endcsname{\let\PY@bf=\textbf\def\PY@tc##1{\textcolor[rgb]{0.60,0.60,0.60}{##1}}}
\expandafter\def\csname PY@tok@na\endcsname{\def\PY@tc##1{\textcolor[rgb]{0.49,0.56,0.16}{##1}}}
\expandafter\def\csname PY@tok@nt\endcsname{\let\PY@bf=\textbf\def\PY@tc##1{\textcolor[rgb]{0.00,0.50,0.00}{##1}}}
\expandafter\def\csname PY@tok@nd\endcsname{\def\PY@tc##1{\textcolor[rgb]{0.67,0.13,1.00}{##1}}}
\expandafter\def\csname PY@tok@s\endcsname{\def\PY@tc##1{\textcolor[rgb]{0.73,0.13,0.13}{##1}}}
\expandafter\def\csname PY@tok@sd\endcsname{\let\PY@it=\textit\def\PY@tc##1{\textcolor[rgb]{0.73,0.13,0.13}{##1}}}
\expandafter\def\csname PY@tok@si\endcsname{\let\PY@bf=\textbf\def\PY@tc##1{\textcolor[rgb]{0.73,0.40,0.53}{##1}}}
\expandafter\def\csname PY@tok@se\endcsname{\let\PY@bf=\textbf\def\PY@tc##1{\textcolor[rgb]{0.73,0.40,0.13}{##1}}}
\expandafter\def\csname PY@tok@sr\endcsname{\def\PY@tc##1{\textcolor[rgb]{0.73,0.40,0.53}{##1}}}
\expandafter\def\csname PY@tok@ss\endcsname{\def\PY@tc##1{\textcolor[rgb]{0.10,0.09,0.49}{##1}}}
\expandafter\def\csname PY@tok@sx\endcsname{\def\PY@tc##1{\textcolor[rgb]{0.00,0.50,0.00}{##1}}}
\expandafter\def\csname PY@tok@m\endcsname{\def\PY@tc##1{\textcolor[rgb]{0.40,0.40,0.40}{##1}}}
\expandafter\def\csname PY@tok@gh\endcsname{\let\PY@bf=\textbf\def\PY@tc##1{\textcolor[rgb]{0.00,0.00,0.50}{##1}}}
\expandafter\def\csname PY@tok@gu\endcsname{\let\PY@bf=\textbf\def\PY@tc##1{\textcolor[rgb]{0.50,0.00,0.50}{##1}}}
\expandafter\def\csname PY@tok@gd\endcsname{\def\PY@tc##1{\textcolor[rgb]{0.63,0.00,0.00}{##1}}}
\expandafter\def\csname PY@tok@gi\endcsname{\def\PY@tc##1{\textcolor[rgb]{0.00,0.63,0.00}{##1}}}
\expandafter\def\csname PY@tok@gr\endcsname{\def\PY@tc##1{\textcolor[rgb]{1.00,0.00,0.00}{##1}}}
\expandafter\def\csname PY@tok@ge\endcsname{\let\PY@it=\textit}
\expandafter\def\csname PY@tok@gs\endcsname{\let\PY@bf=\textbf}
\expandafter\def\csname PY@tok@gp\endcsname{\let\PY@bf=\textbf\def\PY@tc##1{\textcolor[rgb]{0.00,0.00,0.50}{##1}}}
\expandafter\def\csname PY@tok@go\endcsname{\def\PY@tc##1{\textcolor[rgb]{0.53,0.53,0.53}{##1}}}
\expandafter\def\csname PY@tok@gt\endcsname{\def\PY@tc##1{\textcolor[rgb]{0.00,0.27,0.87}{##1}}}
\expandafter\def\csname PY@tok@err\endcsname{\def\PY@bc##1{\setlength{\fboxsep}{0pt}\fcolorbox[rgb]{1.00,0.00,0.00}{1,1,1}{\strut ##1}}}
\expandafter\def\csname PY@tok@kc\endcsname{\let\PY@bf=\textbf\def\PY@tc##1{\textcolor[rgb]{0.00,0.50,0.00}{##1}}}
\expandafter\def\csname PY@tok@kd\endcsname{\let\PY@bf=\textbf\def\PY@tc##1{\textcolor[rgb]{0.00,0.50,0.00}{##1}}}
\expandafter\def\csname PY@tok@kn\endcsname{\let\PY@bf=\textbf\def\PY@tc##1{\textcolor[rgb]{0.00,0.50,0.00}{##1}}}
\expandafter\def\csname PY@tok@kr\endcsname{\let\PY@bf=\textbf\def\PY@tc##1{\textcolor[rgb]{0.00,0.50,0.00}{##1}}}
\expandafter\def\csname PY@tok@bp\endcsname{\def\PY@tc##1{\textcolor[rgb]{0.00,0.50,0.00}{##1}}}
\expandafter\def\csname PY@tok@fm\endcsname{\def\PY@tc##1{\textcolor[rgb]{0.00,0.00,1.00}{##1}}}
\expandafter\def\csname PY@tok@vc\endcsname{\def\PY@tc##1{\textcolor[rgb]{0.10,0.09,0.49}{##1}}}
\expandafter\def\csname PY@tok@vg\endcsname{\def\PY@tc##1{\textcolor[rgb]{0.10,0.09,0.49}{##1}}}
\expandafter\def\csname PY@tok@vi\endcsname{\def\PY@tc##1{\textcolor[rgb]{0.10,0.09,0.49}{##1}}}
\expandafter\def\csname PY@tok@vm\endcsname{\def\PY@tc##1{\textcolor[rgb]{0.10,0.09,0.49}{##1}}}
\expandafter\def\csname PY@tok@sa\endcsname{\def\PY@tc##1{\textcolor[rgb]{0.73,0.13,0.13}{##1}}}
\expandafter\def\csname PY@tok@sb\endcsname{\def\PY@tc##1{\textcolor[rgb]{0.73,0.13,0.13}{##1}}}
\expandafter\def\csname PY@tok@sc\endcsname{\def\PY@tc##1{\textcolor[rgb]{0.73,0.13,0.13}{##1}}}
\expandafter\def\csname PY@tok@dl\endcsname{\def\PY@tc##1{\textcolor[rgb]{0.73,0.13,0.13}{##1}}}
\expandafter\def\csname PY@tok@s2\endcsname{\def\PY@tc##1{\textcolor[rgb]{0.73,0.13,0.13}{##1}}}
\expandafter\def\csname PY@tok@sh\endcsname{\def\PY@tc##1{\textcolor[rgb]{0.73,0.13,0.13}{##1}}}
\expandafter\def\csname PY@tok@s1\endcsname{\def\PY@tc##1{\textcolor[rgb]{0.73,0.13,0.13}{##1}}}
\expandafter\def\csname PY@tok@mb\endcsname{\def\PY@tc##1{\textcolor[rgb]{0.40,0.40,0.40}{##1}}}
\expandafter\def\csname PY@tok@mf\endcsname{\def\PY@tc##1{\textcolor[rgb]{0.40,0.40,0.40}{##1}}}
\expandafter\def\csname PY@tok@mh\endcsname{\def\PY@tc##1{\textcolor[rgb]{0.40,0.40,0.40}{##1}}}
\expandafter\def\csname PY@tok@mi\endcsname{\def\PY@tc##1{\textcolor[rgb]{0.40,0.40,0.40}{##1}}}
\expandafter\def\csname PY@tok@il\endcsname{\def\PY@tc##1{\textcolor[rgb]{0.40,0.40,0.40}{##1}}}
\expandafter\def\csname PY@tok@mo\endcsname{\def\PY@tc##1{\textcolor[rgb]{0.40,0.40,0.40}{##1}}}
\expandafter\def\csname PY@tok@ch\endcsname{\let\PY@it=\textit\def\PY@tc##1{\textcolor[rgb]{0.25,0.50,0.50}{##1}}}
\expandafter\def\csname PY@tok@cm\endcsname{\let\PY@it=\textit\def\PY@tc##1{\textcolor[rgb]{0.25,0.50,0.50}{##1}}}
\expandafter\def\csname PY@tok@cpf\endcsname{\let\PY@it=\textit\def\PY@tc##1{\textcolor[rgb]{0.25,0.50,0.50}{##1}}}
\expandafter\def\csname PY@tok@c1\endcsname{\let\PY@it=\textit\def\PY@tc##1{\textcolor[rgb]{0.25,0.50,0.50}{##1}}}
\expandafter\def\csname PY@tok@cs\endcsname{\let\PY@it=\textit\def\PY@tc##1{\textcolor[rgb]{0.25,0.50,0.50}{##1}}}

\def\PYZbs{\char`\\}
\def\PYZus{\char`\_}
\def\PYZob{\char`\{}
\def\PYZcb{\char`\}}
\def\PYZca{\char`\^}
\def\PYZam{\char`\&}
\def\PYZlt{\char`\<}
\def\PYZgt{\char`\>}
\def\PYZsh{\char`\#}
\def\PYZpc{\char`\%}
\def\PYZdl{\char`\$}
\def\PYZhy{\char`\-}
\def\PYZsq{\char`\'}
\def\PYZdq{\char`\"}
\def\PYZti{\char`\~}
% for compatibility with earlier versions
\def\PYZat{@}
\def\PYZlb{[}
\def\PYZrb{]}
\makeatother


    % Exact colors from NB
    \definecolor{incolor}{rgb}{0.0, 0.0, 0.5}
    \definecolor{outcolor}{rgb}{0.545, 0.0, 0.0}



    
    % Prevent overflowing lines due to hard-to-break entities
    \sloppy 
    % Setup hyperref package
    \hypersetup{
      breaklinks=true,  % so long urls are correctly broken across lines
      colorlinks=true,
      urlcolor=urlcolor,
      linkcolor=linkcolor,
      citecolor=citecolor,
      }
    % Slightly bigger margins than the latex defaults
    
    \geometry{verbose,tmargin=1in,bmargin=1in,lmargin=1in,rmargin=1in}
    
    

    \begin{document}
    
    
    \maketitle
    
    

    
    \hypertarget{conclusion}{%
\section{Conclusion}\label{conclusion}}

As shown in the above dataframe:

\begin{enumerate}
\def\labelenumi{\arabic{enumi}.}
\tightlist
\item
  when lambda keeps the same, the variance of L increase as n increases.
\item
  when n keeps the same, the variance of L increase as lambda increases.
\end{enumerate}

when other factors remains the same, the the stability of the system
decreases as n or lambda increase. When the arrival interval is less
than the service time, the system is much more unstable than the case
that arrival interval is longer than the service time. The customers
accumulated much faster in the queue for system where the arrival
interval is shorter than the service time.

 

    \begin{Verbatim}[commandchars=\\\{\}]
{\color{incolor}In [{\color{incolor}95}]:} \PY{k+kn}{from} \PY{n+nn}{mytypes} \PY{k}{import} \PY{n}{Queue}
         \PY{k+kn}{import} \PY{n+nn}{random}
         \PY{k+kn}{import} \PY{n+nn}{numpy} \PY{k}{as} \PY{n+nn}{np}
         \PY{k+kn}{import} \PY{n+nn}{pandas} \PY{k}{as} \PY{n+nn}{pd}
         \PY{k+kn}{import} \PY{n+nn}{matplotlib} \PY{k}{as} \PY{n+nn}{mp}
\end{Verbatim}


    \begin{Verbatim}[commandchars=\\\{\}]
{\color{incolor}In [{\color{incolor} }]:} \PY{c+c1}{\PYZsh{} A class to represent a single customer in an M/D/1 queue simulation.}
        \PY{c+c1}{\PYZsh{} Each customer has three attributes:}
        \PY{c+c1}{\PYZsh{}}
        \PY{c+c1}{\PYZsh{}  \PYZhy{} cid: A customer identifier (can be anything, but we will use consecutive integers)}
        \PY{c+c1}{\PYZsh{}  \PYZhy{} arrival\PYZus{}time: The time at which the customer arrived at the queue}
        \PY{c+c1}{\PYZsh{}  \PYZhy{} departure\PYZus{}time: The time at which the customer departed the queue}
        \PY{k}{class} \PY{n+nc}{Customer}\PY{p}{(}\PY{n+nb}{object}\PY{p}{)}\PY{p}{:}
            \PY{n}{CUSTOMER\PYZus{}ID} \PY{o}{=} \PY{l+m+mi}{0}
        
            \PY{k}{def} \PY{n+nf}{\PYZus{}\PYZus{}init\PYZus{}\PYZus{}}\PY{p}{(}\PY{n+nb+bp}{self}\PY{p}{,} \PY{n}{arrival\PYZus{}time}\PY{p}{)}\PY{p}{:}
                \PY{n}{Customer}\PY{o}{.}\PY{n}{CUSTOMER\PYZus{}ID} \PY{o}{+}\PY{o}{=} \PY{l+m+mi}{1}
                \PY{n+nb+bp}{self}\PY{o}{.}\PY{n}{cid} \PY{o}{=} \PY{n}{Customer}\PY{o}{.}\PY{n}{CUSTOMER\PYZus{}ID}
                \PY{n+nb+bp}{self}\PY{o}{.}\PY{n}{arrival\PYZus{}time} \PY{o}{=} \PY{n}{arrival\PYZus{}time}
                \PY{n+nb+bp}{self}\PY{o}{.}\PY{n}{departure\PYZus{}time} \PY{o}{=} \PY{k+kc}{None}
                
            \PY{n+nd}{@property}
            \PY{k}{def} \PY{n+nf}{wait}\PY{p}{(}\PY{n+nb+bp}{self}\PY{p}{)}\PY{p}{:}
                \PY{k}{if} \PY{n+nb+bp}{self}\PY{o}{.}\PY{n}{departure\PYZus{}time} \PY{o+ow}{is} \PY{k+kc}{None}\PY{p}{:}
                    \PY{k}{return} \PY{k+kc}{None}
                \PY{k}{else}\PY{p}{:}
                    \PY{k}{return} \PY{n+nb+bp}{self}\PY{o}{.}\PY{n}{departure\PYZus{}time} \PY{o}{\PYZhy{}} \PY{n+nb+bp}{self}\PY{o}{.}\PY{n}{arrival\PYZus{}time}
                
            \PY{k}{def} \PY{n+nf}{\PYZus{}\PYZus{}str\PYZus{}\PYZus{}}\PY{p}{(}\PY{n+nb+bp}{self}\PY{p}{)}\PY{p}{:}
                \PY{k}{return} \PY{l+s+s2}{\PYZdq{}}\PY{l+s+s2}{Customer(}\PY{l+s+si}{\PYZob{}\PYZcb{}}\PY{l+s+s2}{, }\PY{l+s+si}{\PYZob{}\PYZcb{}}\PY{l+s+s2}{)}\PY{l+s+s2}{\PYZdq{}}\PY{o}{.}\PY{n}{format}\PY{p}{(}\PY{n+nb+bp}{self}\PY{o}{.}\PY{n}{cid}\PY{p}{,} \PY{n+nb+bp}{self}\PY{o}{.}\PY{n}{arrival\PYZus{}time}\PY{p}{)}
            
            \PY{k}{def} \PY{n+nf}{\PYZus{}\PYZus{}repr\PYZus{}\PYZus{}}\PY{p}{(}\PY{n+nb+bp}{self}\PY{p}{)}\PY{p}{:}
                \PY{k}{return} \PY{n+nb}{str}\PY{p}{(}\PY{n+nb+bp}{self}\PY{p}{)}
\end{Verbatim}


    \begin{Verbatim}[commandchars=\\\{\}]
{\color{incolor}In [{\color{incolor}49}]:} \PY{c+c1}{\PYZsh{} simulate\PYZus{}md1: Simulates an M/D/1 queue.}
         \PY{c+c1}{\PYZsh{}}
         \PY{c+c1}{\PYZsh{} In an M/D/1 queue que have:}
         \PY{c+c1}{\PYZsh{}   }
         \PY{c+c1}{\PYZsh{} \PYZhy{} Arrivals follow a Markov process (M)}
         \PY{c+c1}{\PYZsh{} \PYZhy{} The time to service each customer is deterministic (D)}
         \PY{c+c1}{\PYZsh{} \PYZhy{} There is only one server (1)}
         \PY{c+c1}{\PYZsh{}}
         \PY{c+c1}{\PYZsh{} The function takes three parameters (plus one optional parameter)}
         \PY{c+c1}{\PYZsh{}}
         \PY{c+c1}{\PYZsh{} \PYZhy{} lambd: The simulation uses an exponential distribution to determine}
         \PY{c+c1}{\PYZsh{}          the arrival time of the next customer. This parameters is the}
         \PY{c+c1}{\PYZsh{}          lambda parameter to an exponential distribution (specifically,}
         \PY{c+c1}{\PYZsh{}          Python\PYZsq{}s random.expovariate)}
         \PY{c+c1}{\PYZsh{} \PYZhy{} mu: The rate at which customers are serviced. The larger this value is,}
         \PY{c+c1}{\PYZsh{}       the more customers will be serviced per unit of time}
         \PY{c+c1}{\PYZsh{} \PYZhy{} max\PYZus{}time: The maximum time of the simulation}
         \PY{c+c1}{\PYZsh{} \PYZhy{} verbosity (optional): Can be 0 (no output), 1 (print state of the queue}
         \PY{c+c1}{\PYZsh{}                         at each time), or 2 (same as 1, but also print when}
         \PY{c+c1}{\PYZsh{}                         each customer arrives and departs)}
         \PY{c+c1}{\PYZsh{}}
         \PY{c+c1}{\PYZsh{} The function returns two lists: one with all the customers that were served}
         \PY{c+c1}{\PYZsh{} during the simulation, and one with all the customers that were yet to be}
         \PY{c+c1}{\PYZsh{} served when the simulation ended.}
         \PY{c+c1}{\PYZsh{}}
         
         \PY{k}{def} \PY{n+nf}{simulate\PYZus{}md1}\PY{p}{(}\PY{n}{lambd}\PY{p}{,} \PY{n}{mu}\PY{p}{,} \PY{n}{max\PYZus{}time}\PY{p}{,} \PY{n}{verbosity} \PY{o}{=} \PY{l+m+mi}{0}\PY{p}{)}\PY{p}{:}
             \PY{n}{md1} \PY{o}{=} \PY{n}{Queue}\PY{p}{(}\PY{p}{)}
             \PY{n}{L\PYZus{}array} \PY{o}{=}\PY{p}{[}\PY{p}{]}
         
             \PY{c+c1}{\PYZsh{} Our return values: the list of customers that have been}
             \PY{c+c1}{\PYZsh{} served, and the list of customers that haven\PYZsq{}t been served}
             \PY{n}{served\PYZus{}customers} \PY{o}{=} \PY{p}{[}\PY{p}{]}
             \PY{n}{unserved\PYZus{}customers} \PY{o}{=} \PY{p}{[}\PY{p}{]}
             
             \PY{c+c1}{\PYZsh{} The type of simulation we have implemented in this function}
             \PY{c+c1}{\PYZsh{} is known as a \PYZdq{}discrete event simulation\PYZdq{}}
             \PY{c+c1}{\PYZsh{} (https://en.wikipedia.org/wiki/Discrete\PYZus{}event\PYZus{}simulation), where}
             \PY{c+c1}{\PYZsh{} we simulate a discrete sequence of events: customer arrivals}
             \PY{c+c1}{\PYZsh{} and customer departures. So, we only need to keep track of when }
             \PY{c+c1}{\PYZsh{} the next arrival and the next departure will take place (because }
             \PY{c+c1}{\PYZsh{} nothing interesting happens between those two types of events). }
             \PY{c+c1}{\PYZsh{} Then, in each step of the simulation, we simply advance the }
             \PY{c+c1}{\PYZsh{} simulation clock to earliest next event. Note that, because}
             \PY{c+c1}{\PYZsh{} we have a single server, this can be easily done with just}
             \PY{c+c1}{\PYZsh{} two variables.}
         
             \PY{n}{next\PYZus{}arrival} \PY{o}{=} \PY{n}{random}\PY{o}{.}\PY{n}{expovariate}\PY{p}{(}\PY{n}{lambd}\PY{p}{)}
             \PY{n}{next\PYZus{}service} \PY{o}{=} \PY{n}{next\PYZus{}arrival} \PY{o}{+} \PY{l+m+mi}{1}\PY{o}{/}\PY{n}{mu}
                 
             \PY{c+c1}{\PYZsh{} We initialize the simulation\PYZsq{}s time to the earliest event:}
             \PY{c+c1}{\PYZsh{} the next arrival time}
             \PY{n}{t} \PY{o}{=} \PY{n}{next\PYZus{}arrival}
             
             \PY{k}{while} \PY{n}{t} \PY{o}{\PYZlt{}} \PY{n}{max\PYZus{}time}\PY{p}{:}
         
                 \PY{c+c1}{\PYZsh{} Process a new arrival}
                 \PY{k}{if} \PY{n}{t} \PY{o}{==} \PY{n}{next\PYZus{}arrival}\PY{p}{:}
                     \PY{n}{customer} \PY{o}{=} \PY{n}{Customer}\PY{p}{(}\PY{n}{arrival\PYZus{}time} \PY{o}{=} \PY{n}{t}\PY{p}{)}
                     \PY{n}{md1}\PY{o}{.}\PY{n}{enqueue}\PY{p}{(}\PY{n}{customer}\PY{p}{)}
         
                     \PY{c+c1}{\PYZsh{}if verbosity \PYZgt{}= 2:}
                         \PY{c+c1}{\PYZsh{}print(\PYZdq{}\PYZob{}:10.2f\PYZcb{}: Customer \PYZob{}\PYZcb{} arrives\PYZdq{}.format(t, customer.cid))}
         
                     \PY{n}{next\PYZus{}arrival} \PY{o}{=} \PY{n}{t} \PY{o}{+} \PY{n}{random}\PY{o}{.}\PY{n}{expovariate}\PY{p}{(}\PY{n}{lambd}\PY{p}{)}
                     
                 \PY{c+c1}{\PYZsh{} The customer at the head of the queue has been served}
                 \PY{k}{if} \PY{n}{t} \PY{o}{==} \PY{n}{next\PYZus{}service}\PY{p}{:}
                     \PY{n}{done\PYZus{}customer} \PY{o}{=} \PY{n}{md1}\PY{o}{.}\PY{n}{dequeue}\PY{p}{(}\PY{p}{)}
                     \PY{n}{done\PYZus{}customer}\PY{o}{.}\PY{n}{departure\PYZus{}time} \PY{o}{=} \PY{n}{t}
                     
                     \PY{n}{served\PYZus{}customers}\PY{o}{.}\PY{n}{append}\PY{p}{(}\PY{n}{done\PYZus{}customer}\PY{p}{)}
         
                     \PY{c+c1}{\PYZsh{}if verbosity \PYZgt{}= 2:}
                         \PY{c+c1}{\PYZsh{}print(\PYZdq{}\PYZob{}:10.2f\PYZcb{}: Customer \PYZob{}\PYZcb{} departs\PYZdq{}.format(t, done\PYZus{}customer.cid))            }
                     
                     \PY{k}{if} \PY{n}{md1}\PY{o}{.}\PY{n}{is\PYZus{}empty}\PY{p}{(}\PY{p}{)}\PY{p}{:}
                         \PY{c+c1}{\PYZsh{} The next service time will be 1/mu after the next arrival}
                         \PY{n}{next\PYZus{}service} \PY{o}{=} \PY{n}{next\PYZus{}arrival} \PY{o}{+} \PY{l+m+mi}{1}\PY{o}{/}\PY{n}{mu}
                     \PY{k}{else}\PY{p}{:}
                         \PY{c+c1}{\PYZsh{} We start serving the next customer, so the next service time}
                         \PY{c+c1}{\PYZsh{} will be 1/mu after the current time.}
                         \PY{n}{next\PYZus{}service} \PY{o}{=} \PY{n}{t} \PY{o}{+} \PY{l+m+mi}{1}\PY{o}{/}\PY{n}{mu}
                     
                 \PY{c+c1}{\PYZsh{}if verbosity \PYZgt{}= 1:}
                     \PY{c+c1}{\PYZsh{}print(\PYZdq{}\PYZob{}:10.2f\PYZcb{}: \PYZob{}\PYZcb{}\PYZdq{}.format(t, \PYZdq{}\PYZsh{}\PYZdq{}*md1.length))}
                     \PY{c+c1}{\PYZsh{}store the L in L\PYZus{}array }
                 \PY{n}{L\PYZus{}array}\PY{o}{.}\PY{n}{append}\PY{p}{(}\PY{n}{md1}\PY{o}{.}\PY{n}{length}\PY{p}{)}
                 \PY{c+c1}{\PYZsh{} Advance the simulation clock to the next event}
                 \PY{n}{t} \PY{o}{=} \PY{n+nb}{min}\PY{p}{(}\PY{n}{next\PYZus{}arrival}\PY{p}{,} \PY{n}{next\PYZus{}service}\PY{p}{)}
                 
             \PY{c+c1}{\PYZsh{} Any remaining customers in the queue haven\PYZsq{}t been served}
             \PY{k}{while} \PY{o+ow}{not} \PY{n}{md1}\PY{o}{.}\PY{n}{is\PYZus{}empty}\PY{p}{(}\PY{p}{)}\PY{p}{:}
                 \PY{n}{unserved\PYZus{}customers}\PY{o}{.}\PY{n}{append}\PY{p}{(}\PY{n}{md1}\PY{o}{.}\PY{n}{dequeue}\PY{p}{(}\PY{p}{)}\PY{p}{)}
                 
             \PY{k}{return} \PY{n}{L\PYZus{}array}
\end{Verbatim}


    mu is service rate instead of service, divide 1 hour per customer by 1
customer per hour --\textgreater{} mu =1

    \begin{Verbatim}[commandchars=\\\{\}]
{\color{incolor}In [{\color{incolor}92}]:} \PY{c+c1}{\PYZsh{}loop through 9 different cases and record the results}
         \PY{n}{result\PYZus{}df} \PY{o}{=} \PY{n}{pd}\PY{o}{.}\PY{n}{DataFrame}\PY{p}{(}\PY{n}{columns} \PY{o}{=} \PY{p}{[}\PY{l+s+s1}{\PYZsq{}}\PY{l+s+s1}{n}\PY{l+s+s1}{\PYZsq{}}\PY{p}{,} \PY{l+s+s1}{\PYZsq{}}\PY{l+s+s1}{lambda}\PY{l+s+s1}{\PYZsq{}}\PY{p}{,} \PY{l+s+s1}{\PYZsq{}}\PY{l+s+s1}{mean}\PY{l+s+s1}{\PYZsq{}}\PY{p}{,} \PY{l+s+s1}{\PYZsq{}}\PY{l+s+s1}{variance}\PY{l+s+s1}{\PYZsq{}}\PY{p}{]}\PY{p}{)}
         \PY{n}{j} \PY{o}{=} \PY{l+m+mi}{0}
         \PY{n}{L\PYZus{}array} \PY{o}{=} \PY{p}{[}\PY{p}{]}
         \PY{n}{full\PYZus{}L\PYZus{}array} \PY{o}{=} \PY{p}{[}\PY{p}{]}
         
         
         \PY{k}{for} \PY{n}{lambd} \PY{o+ow}{in} \PY{p}{[}\PY{l+m+mf}{0.9}\PY{p}{,} \PY{l+m+mi}{1}\PY{p}{,} \PY{l+m+mf}{1.1}\PY{p}{]}\PY{p}{:}
             \PY{k}{for} \PY{n}{n} \PY{o+ow}{in} \PY{p}{[}\PY{l+m+mi}{100}\PY{p}{,} \PY{l+m+mi}{1000}\PY{p}{,} \PY{l+m+mi}{10000}\PY{p}{]}\PY{p}{:}
                 \PY{n}{L\PYZus{}array} \PY{o}{=} \PY{p}{[}\PY{p}{]}
                 \PY{n}{full\PYZus{}L\PYZus{}array} \PY{o}{=} \PY{p}{[}\PY{p}{]}
                 \PY{k}{for} \PY{n}{i} \PY{o+ow}{in} \PY{n+nb}{range}\PY{p}{(}\PY{l+m+mi}{10}\PY{p}{)}\PY{p}{:}
                     \PY{n}{L\PYZus{}array} \PY{o}{=} \PY{n}{simulate\PYZus{}md1}\PY{p}{(}\PY{n}{lambd}\PY{p}{,} \PY{l+m+mi}{1}\PY{p}{,} \PY{n}{n}\PY{p}{,} \PY{n}{verbosity}\PY{o}{=}\PY{l+m+mi}{2}\PY{p}{)}
                     \PY{n}{full\PYZus{}L\PYZus{}array}\PY{o}{.}\PY{n}{extend}\PY{p}{(}\PY{n}{L\PYZus{}array}\PY{p}{)}
                 \PY{n}{mean\PYZus{}L} \PY{o}{=} \PY{n}{np}\PY{o}{.}\PY{n}{mean}\PY{p}{(}\PY{n}{full\PYZus{}L\PYZus{}array}\PY{p}{)}
                 \PY{n}{var\PYZus{}L} \PY{o}{=} \PY{n}{np}\PY{o}{.}\PY{n}{var}\PY{p}{(}\PY{n}{full\PYZus{}L\PYZus{}array}\PY{p}{)}
                 \PY{n}{rows} \PY{o}{=} \PY{p}{[}\PY{n}{n}\PY{p}{,} \PY{n}{lambd}\PY{p}{,} \PY{n}{mean\PYZus{}L}\PY{p}{,} \PY{n}{var\PYZus{}L}\PY{p}{]}
                 \PY{n}{result\PYZus{}df}\PY{o}{.}\PY{n}{loc}\PY{p}{[}\PY{n}{j}\PY{p}{]} \PY{o}{=} \PY{n}{rows}
                 \PY{n+nb}{print}\PY{p}{(}\PY{n+nb}{str}\PY{p}{(}\PY{n}{n}\PY{p}{)}\PY{p}{,} \PY{n+nb}{str}\PY{p}{(}\PY{n}{lambd}\PY{p}{)}\PY{p}{,} \PY{n}{mean\PYZus{}L}\PY{p}{,} \PY{n}{var\PYZus{}L}\PY{p}{)}
                 \PY{n}{j}\PY{o}{+}\PY{o}{=}\PY{l+m+mi}{1}
\end{Verbatim}


    \begin{Verbatim}[commandchars=\\\{\}]
100 0.9 3.27745995423 5.61924023009
1000 0.9 5.8154414196 24.0671725622
10000 0.9 5.25399388501 20.1635561136
100 1 6.28780743066 18.3556807499
1000 1 11.3988602526 63.2809454101
10000 1 49.3329969525 1358.5442825
100 1.1 10.7873345936 42.8942819315
1000 1.1 55.1967848428 1266.39747495
10000 1.1 493.967715469 82924.903268

    \end{Verbatim}

    \begin{Verbatim}[commandchars=\\\{\}]
{\color{incolor}In [{\color{incolor} }]:} \PY{n+nb}{print}\PY{p}{(}\PY{n}{full\PYZus{}L\PYZus{}array}\PY{p}{)}
\end{Verbatim}


    \begin{Verbatim}[commandchars=\\\{\}]
{\color{incolor}In [{\color{incolor}94}]:} \PY{n}{result\PYZus{}df}
\end{Verbatim}


\begin{Verbatim}[commandchars=\\\{\}]
{\color{outcolor}Out[{\color{outcolor}94}]:}          n  lambda        mean      variance
         0    100.0     0.9    3.277460      5.619240
         1   1000.0     0.9    5.815441     24.067173
         2  10000.0     0.9    5.253994     20.163556
         3    100.0     1.0    6.287807     18.355681
         4   1000.0     1.0   11.398860     63.280945
         5  10000.0     1.0   49.332997   1358.544283
         6    100.0     1.1   10.787335     42.894282
         7   1000.0     1.1   55.196785   1266.397475
         8  10000.0     1.1  493.967715  82924.903268
\end{Verbatim}
            
    \hypertarget{conclusion}{%
\section{Conclusion}\label{conclusion}}

As shown in the above dataframe:

\begin{enumerate}
\def\labelenumi{\arabic{enumi}.}
\tightlist
\item
  when lambda keeps the same, the variance of L increase as n increases.
\item
  when n keeps the same, the variance of L increase as lambda increases.
\end{enumerate}

when other factors remains the same, the the stability of the system
decreases as n or lambda increase. When the arrival interval is less
than the service time, the system is much more unstable than the case
that arrival interval is longer than the service time. The customers
accumulated much faster in the queue for system where the arrival
interval is shorter than the service time.

 

    \hypertarget{reference}{%
\paragraph{Reference}\label{reference}}

The code above is adapted from a lectrure example for CMSC 12100/CAPP
30121 at University of Chicago. The course was instructed by multiple
instructors and teaching assistants.

The code above is modified for the use of calculating average system
length / average customers in the system (L).

https://classes.cs.uchicago.edu/archive/2017/fall/12100-1/lecture-examples/


    % Add a bibliography block to the postdoc
    
    
    
    \end{document}
